\documentclass[thesis]{thesis}
\usepackage{mypackage}

\title{\LaTeX中文报告}
\author{佚名}
\school{(XXX学院,XXX专业,学号)}
\authorremark{Email: \url{xxx@mail.edu.cn}}

% \linespread{1.15}

\begin{document}

\maketitle

\begin{abstractchinese}
这是\LaTeX论文的中文测试摘要。句子要长长长长长长长非常长,以便展示多行间距。
\end{abstractchinese}

\begin{keywordchinese}
\LaTeX文档,中文论文
\end{keywordchinese}

% \xiaosi

这是测试文档内容,默认为五号宋体,行间距为单倍行距。
如果需要改为小四宋体,可以直接在文档前面添加\verb'\xiaosi'来修改字号。
如果需要修改行间距,则修改文档引言区的\verb'\linespread'参数。

下面是测试引用,参考文献内容应符合中文文献规范,如期刊\cite{journal}、专著\cite{book}、论文集\cite{thesis}、报告\cite{report}、电子文献\cite{database,announcement}。


\begin{thebibliography}{1}
\bibitem{journal} 期刊作者. 题名[J]. 刊名. 出版年,卷(期): 起止页码.

\bibitem{book} 专著作者. 书名[M]. 版次(第一版可略). 出版地: 出版社,出版年: 起止页码.

\bibitem{thesis} 论文集作者. 题名[C]. 编者. 论文集名. 出版地: 出版社,出版年: 起止页码.

\bibitem{report} 报告作者.题名[R]. 报告地: 报告会主办单位,年份.

\bibitem{database} 电子文献作者. 题名[DB/OL]. 数据库,日期.

\bibitem{announcement} 电子文献作者. 题名[EB/OL]. 网上电子公告,日期.

\end{thebibliography}

%% with reference.bib file
% \bibliographystyle{unsrt}
% \bibliography{reference}

\end{document}